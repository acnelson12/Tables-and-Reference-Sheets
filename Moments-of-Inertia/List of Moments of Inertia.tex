\documentclass{article}
\usepackage[latin1]{inputenc}
\usepackage{amsmath}
\usepackage{amsfonts}
\usepackage{amssymb}
\usepackage[paperwidth=5.5in, paperheight=8.5in, left=0in, right=0in, top=0.5in]{geometry}
\usepackage{nopageno}

\DeclareMathOperator{\dif}{d \!} % Differential operator

\begin{document}

\begin{center}
\begin{LARGE}
List of Moments of Inertia
\end{LARGE}
\end{center}

$$
I = \int_m r^2 \dif m
$$

\begin{tabular}{p{1.75in}|p{2in}|l}
\hline
\textbf{Body} & \textbf{Axis} & \textbf{Moment of} \\
& & \textbf{Inertia} \\
\hline
\hline & &\\
Uniform thin rod of length $l$ & Normal to the length, at one end & $\frac{1}{3} m l^2$ \\
& &\\
Uniform thin rod of length $l$ & Normal to the length, at the center & $\frac{1}{12} m l^2$ \\

Thin rectangular sheet, sides $a$ and $b$ & Through the center parallel to $b$ & $\frac{1}{12} m a^2$ \\
Thin rectangular sheet, sides $a$ and $b$ & Through the center perpendicular to the sheet & $\frac{1}{12} m \left(a^2 + b^2\right)$ \\
Disk of radius $r$ & Normal to the disk through the center & $\frac{1}{2}mr^2$ \\
Disk of radius $r$ & Along any diameter & $\frac{1}{4}mr^2$ \\
& &\\
Thin circular ring, radii $r_1$ and $r_2$ & Through center normal to plane of ring & $\frac{1}{2}m\left(r_1^2 + r_2^2\right)$ \\
Thin circular ring, radii $r_1$ and $r_2$ & Any diameter & $\frac{1}{4}m\left(r_1^2 + r_2^2\right)$ \\
Rectangular parallelepiped, edges $a$, $b$, and $c$ & Through center perpendicular to face $ab$, (parallel to edge $c$) & $\frac{1}{12}m\left(a^2 + b^2\right)$ \\
Ball, radius $r$ & Any diameter & $\frac{2}{5}mr^2$ \\
& &\\
Sphere, radius $r$ & Any diameter & $\frac{2}{3}mr^2$ \\
& &\\
Spherical shell, external radius $r_1$, internal radius $r_2$ & Any diameter & $\frac{2}{5}m\frac{\left(r_1^5 - r_2^5\right)}{\left(r_1^3 - r_2^3\right)}$ \\
Right circular cylinder of radius $r$, length $l$ & The longitudinal axis & $\frac{1}{2}mr^2$ \\
Right circular cylinder of radius $r$, length $l$ & The transverse diameter & $m\left(\frac{r^2}{4} + \frac{l^2}{12}\right)$ \\
Hollow circular cylinder, length $l$, radii $r_1$ and $r_2$ & The longitudinal axis & $\frac{1}{2}m\left(r_1^2 + r_2^2\right)$ \\
Hollow thin circular cylinder, length $l$, radius $r$ & Transverse diameter & $m\left(\frac{r^2}{2} + \frac{l^2}{12}\right)$ \\
\end{tabular}

\end{document}
