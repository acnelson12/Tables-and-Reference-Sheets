\documentclass{article}
\usepackage[utf8]{inputenc}
\usepackage{amsmath}
\usepackage{esint}
\usepackage[paperwidth=5.5in, paperheight=8.5in]{geometry}

% Bold italic vectors
\newcommand{\vect}[1]{\boldsymbol{#1}}
% Differential (upface d)
\DeclareMathOperator{\dif}{d \!}

\begin{document}

\section*{Maxwell's Equations}

\subsection*{Gauss's Law}
The electric field leaving a volume is proportional to the charge inside.
$$ \oiint_{\partial \Omega} \vect{E} \cdot \dif \vect{S}
= \frac{1}{\varepsilon_0}\iiint_\Omega \rho \dif V $$
$$ \nabla \cdot \vect{E} = \frac{\rho}{\varepsilon_0}$$

\subsection*{Gauss's Law for Magnetism}
There are no magnetic monopoles; the total magnetic flux piercing a closed surface is zero.
$$ \oiint_{\partial \Omega} \vect{B} \cdot \dif \vect{S} = 0 $$
$$ \nabla \cdot \vect{B} = 0 $$

\subsection*{Maxwell-Faraday Equation \\ (Faraday's Law of Induction)}
The voltage accumulated around a closed circuit is proportional to the time rate of change of the magnetic flux it encloses.
$$ \oint_{\partial \Sigma} \vect{E} \cdot \dif \vect{\ell}
= -\frac{\dif}{\dif t}\iint_\Sigma \vect{B} \cdot \dif \vect{S} $$
$$ \nabla \times \vect{E} = -\frac{\partial \vect{B}}{\partial t} $$

\begin{samepage}
\subsection*{Amp\`ere's Circuital Law \\ (with Maxwell's addition)}
Electric currents and changes in electric fields are proportional to the magnetic field circulating about the area they pierce.
$$ \oint_{\partial \Sigma} \vect{B} \cdot \dif \vect{\ell}
= \mu_0 \iint_\Sigma \vect{J} \cdot \dif \vect{S} +
\mu_0 \varepsilon_0 \frac{\dif}{\dif t}\iint_\Sigma \vect{E}
\cdot \dif \vect{S} $$
$$ \nabla \times \vect{B} = \mu_0 \left( \vect{J} + \varepsilon_0 \frac{\partial \vect{E}}{\partial t} \right)$$
\end{samepage}

\section*{Lorentz Force Law}
If a particle of charge $q$ moves with velocity $\vect{v}$ in the presence of an electric field $\vect{E}$ and a magnetic field $\vect{B}$, then it will experience a force
$$ \vect{F} = q\left[\vect{E}+\vect{v}\times\vect{B}\right]. $$

\end{document}
